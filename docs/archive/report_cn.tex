\documentclass[UTF8,a4paper,12pt]{ctexart}
\usepackage[left=2.5cm,right=2.5cm,top=3cm,bottom=3cm]{geometry}
\usepackage{hyperref}
\usepackage{graphicx}
\usepackage{amsmath}
\usepackage{amssymb}
\usepackage{listings}
\usepackage{xcolor}
\usepackage{float}

% 设置超链接颜色
\hypersetup{
    colorlinks=true,
    linkcolor=blue,
    filecolor=magenta,      
    urlcolor=cyan,
}

\title{\textbf{不确定性量化下的物理信息深度学习综合库}\ \large B-PINNs 项目报告}
\author{作者: Giulia Mescolini, Luca Sosta \ 导师: Prof. Andrea Manzoni, Prof. Stefano Pagani}
\date{2025年11月24日}

\begin{document}

\maketitle

\begin{abstract}
本项目开发了一个新的库,实现了贝叶斯物理信息神经网络(B-PINNs)方法的多种变体。该框架能够利用神经网络解决物理问题,同时考虑来自偏微分方程的信息并量化预测的不确定性。

为了实施这一策略,我们从头开始设计了一个模块化且灵活的管道。在设计中,我们一方面希望将 B-PINN 框架中涉及的网络任务的特性反映到类的层次结构中,另一方面旨在提出一种结构,能够在其中插入针对不同任务的各种功能。

在发布的版本中,我们提供了整个流程各个方面的不同变体,从数据管理到训练算法。其中,我们实现了 Adam、哈密顿蒙特卡洛 (Hamiltonian Monte Carlo, HMC)、变分推断 (Variational Inference, VI) 和斯坦变分梯度下降 (Stein Variational Gradient Descent, SVGD)。

该库的设计也预留了进一步扩展的空间,因为代表不同组件的模块被设计为由少量特定案例的行为定义,并且对功能没有严格的限制。这一点在实现具有不同例程和结构或不同问题的算法时得到了验证。

最后,我们展示了一系列结果,以突显该方法和库的主要特征:算法比较、在神经网络中引入 PDE 残差、高维度的可移植性以及通过微调模型参数可以获得的结果质量。

\textbf{关键词:} B-PINNs,不确定性量化,科学学习
\end{abstract}

\tableofcontents
\newpage

\section{简介 (Introduction)}

\subsection{项目概述}
机器学习技术在科学计算中的整合如今变得越来越流行。一个具有挑战性的场景是不确定性量化,神经网络的高效性有助于克服基于 PDE 求解器的传统方法的计算成本。

贝叶斯物理信息神经网络(B-PINNs)代表了一种解决此类问题的方法,这不仅归功于在学习过程中包含了微分方程的残差,还归功于能够重建网络输出分布而非单一预测的训练算法。

该方法是物理信息神经网络(PINNs)的演变,PINNs 是一种用于解决涉及偏微分方程问题的深度学习框架。B-PINNs 从贝叶斯统计理论中汲取灵感,增强了 PINNs 方法并使得引入不确定性量化成为可能。

在这个项目中,我们实现了一个用于贝叶斯物理信息机器学习的库,旨在实现模块化、灵活性并支持插件。通常只能找到应用于特定微分问题的特定贝叶斯训练方法的实现;因此,我们设定的目标是开发一个具有广泛功能的库。

\subsection{报告结构}
本报告介绍了拟议实现的理论基础:
\begin{itemize}
    \item \textbf{第2章 方法}:解释 B-PINN 的构建模块,从神经网络基础到 PINNs 和 BNNs 的理论。
    \item \textbf{第3章 算法}:深入探讨实现的非确定性训练算法:HMC, VI, SVGD。
    \item \textbf{第4章 代码概述}:介绍环境、库的选择以及面向对象编程的特性。
    \item \textbf{第5章 源代码}:详细说明源代码结构。
    \item \textbf{第6章 结果}:展示库在不同数据集和训练配置下的应用结果。
    \item \textbf{第7章 结论}:总结项目并提出未来发展建议。
\end{itemize}

\section{方法 (Methods)}

\subsection{神经网络概述}
人工神经网络 (NN) 是受生物神经网络启发的计算系统。在深度学习框架下,它们能够处理复杂信息。主要结构包括输入层、隐藏层(特征学习)和输出层。
\begin{equation}
a^{(l)}_i = \phi \left( \sum_{j=1}^{N_{l-1}} w_{j,i} a^{(l-1)}_j + b^{(l)}_i \right)
\end{equation}
常见的激活函数包括 Sigmoid, Tanh, ReLU, Leaky ReLU 和 Swish。本项目采用了全连接前馈神经网络。

\subsubsection{反向传播与优化器}
学习过程通过反向传播算法计算损失函数的梯度,并更新权重。
常见的优化器包括:
\begin{itemize}
    \item \textbf{梯度下降 (GD)}:沿着梯度的反方向更新。
    \item \textbf{随机梯度下降 (SGD)}:使用小批量 (batch) 数据进行更新。
    \item \textbf{Adam}:一种自适应学习率方法,结合了 AdaGrad 和 RMSProp 的优点。
\end{itemize}

\subsection{物理信息神经网络 (PINNs)}
PINNs 是训练用于解决受微分方程控制的监督学习任务的神经网络。
\begin{equation}
\mathcal{N}(u(x, t); \lambda) = f(x, t) \quad \text{in } \Omega \times [0, T]
\end{equation}
PINN 的损失函数包含数据拟合项和 PDE 残差项:
\begin{equation}
LOSS = MSE_{data} + MSE_{residual}
\end{equation}

\subsection{贝叶斯神经网络 (BNNs)}
BNNs 将贝叶斯推理与神经网络结合,引入不确定性量化。
核心思想是用概率分布替换确定的网络参数 $\theta$。根据贝叶斯定理:
\begin{equation}
p(\theta | D) \propto p(D | \theta) p(\theta)
\end{equation}
其中 $p(\theta | D)$ 是后验分布,$p(D | \theta)$ 是似然,$p(\theta)$ 是先验。

\subsection{贝叶斯物理信息神经网络 (B-PINNs)}
B-PINNs 将 PINNs 集成到贝叶斯框架中。PDE 约束作为似然项的附加部分出现:
\begin{equation}
p(\theta | D, R) \propto p(D, R | \theta) p(\theta)
\end{equation}
其中 $R$ 代表 PDE 残差。

\section{算法 (Algorithms)}

\subsection{哈密顿蒙特卡洛 (HMC)}
HMC 是一种马尔可夫链蒙特卡洛 (MCMC) 方法,利用哈密顿动力学来生成符合目标分布的样本。
它引入辅助动量变量 $\mathbf{r}$,构造哈密顿量 $H(\theta, \mathbf{r}) = U(\theta) + \frac{1}{2}\mathbf{r}^T M^{-1} \mathbf{r}$,其中 $U(\theta) = - \log(p(\theta|D))$。
算法包含 Leap-Frog 积分步骤和 Metropolis-Hastings 接受-拒绝步骤。

\subsection{变分推断 (VI)}
VI 旨在通过在参数族中寻找最佳近似来逼近后验分布,从而将采样问题转化为确定性优化问题。
目标是最小化 KL 散度 $D_{KL}(Q(\theta|\zeta) || P(\theta|D))$。

\subsection{斯坦变分梯度下降 (SVGD)}
SVGD 是一种基于粒子的变分推断方法。它使用一组粒子 ${\theta_i}_{i=1}^N$ 来近似目标分布,并通过最小化 KL 散度的函数梯度流来迭代更新粒子。
更新规则涉及核函数 $k(\theta, \theta')$(如 RBF 核),使得粒子在向高概率区域移动的同时保持多样性(排斥力)。

\section{代码概述 (Code Overview)}

\subsection{工作环境}
代码基于 Python 3.10 开发,使用 `virtualenv` 管理环境。
依赖库包括:
\begin{itemize}
    \item \textbf{NumPy, SciPy}: 科学计算。
    \item \textbf{Matplotlib}: 可视化。
    \item \textbf{TensorFlow 2.9.1}: 深度学习框架和自动微分。
\end{itemize}

\subsection{面向对象特性}
代码广泛使用了 Python 的 OOP 特性,如继承(单继承和多继承)、抽象基类 (ABC)、数据类 (dataclass)、迭代器和属性装饰器 (@property),以构建模块化和可扩展的架构。

\subsection{仓库结构}
\begin{itemize}
    \item \texttt{config/}: 包含 .json 配置文件。
    \item \texttt{data/}: 存储生成的数据集 (.npy)。
    \item \texttt{outs/}: 存储实验结果(日志、图表、参数)。
    \item \texttt{src/}: 源代码目录。
\end{itemize}

\section{源代码 (Source Code)}

\subsection{主要模块}
\begin{itemize}
    \item \textbf{main.py}: 主执行脚本,负责参数处理、数据加载/生成、模型构建、训练、评估和绘图。
    \item \textbf{setup/}: 参数处理 (\texttt{Param}) 和数据生成 (\texttt{DataGenerator})。
    \item \textbf{equations/}: 定义微分算子 (\texttt{Operators}) 和 PDE 问题(如 \texttt{Laplace}, \texttt{Oscillator})。
    \item \textbf{networks/}: 定义网络架构。
        \begin{itemize}
            \item \texttt{CoreNN}: 基础神经网络。
            \item \texttt{PhysNN}: 引入物理信息。
            \item \texttt{BayesNN}: 贝叶斯功能的封装。
            \item \texttt{Theta}: 处理网络参数的代数运算。
        \end{itemize}
    \item \textbf{algorithms/}: 实现训练算法。
        \begin{itemize}
            \item \texttt{Algorithm}: 抽象基类。
            \item \texttt{ADAM}, \texttt{HMC}, \texttt{VI}, \texttt{SVGD}: 具体实现。
        \end{itemize}
    \item \textbf{postprocessing/}: 结果存储 (\texttt{Storage}) 和绘图 (\texttt{Plotter})。
\end{itemize}

\section{结果 (Results)}

\subsection{回归问题}
使用 Adam, HMC, SVGD, VI 对简单函数(如 $u(x)=\cos(8x)$)进行回归。HMC 和 SVGD 提供了合理的置信区间。

\subsection{阻尼谐振子问题}
比较了标准 NN、PINN 和 B-PINN。在数据缺失的区域,物理信息的引入显著改善了预测,B-PINN 提供了不确定性估计。

\subsection{高维域}
展示了 2D 拉普拉斯问题的求解结果,验证了代码在多维情况下的可移植性。

\subsection{HMC 展示}
对 HMC 算法进行了深入的参数微调,展示了其在回归和谐振子问题上的高质量结果,特别是在噪声数据下的鲁棒性。

\section{结论 (Conclusions)}
本项目成功实现了一个用于贝叶斯物理信息深度学习的综合库。
\begin{itemize}
    \item \textbf{模块化设计}:基于 OOP 的架构使得扩展新算法和新问题变得容易。
    \item \textbf{多样化算法}:实现了从确定性 (Adam) 到多种贝叶斯 (HMC, VI, SVGD) 的训练方法。
    \item \textbf{物理信息}:验证了 PDE 残差在数据稀缺情况下对模型性能的提升。
    \item \textbf{不确定性量化}:提供了对预测结果置信度的有效估计。
\end{itemize}
未来工作可以包括参数估计问题的扩展(逆问题)、更高级的超参数调优自动化以及对高维数据可视化的改进。

\appendix
\section{安装指南}
\begin{lstlisting}[language=bash]
# 创建虚拟环境
virtualenv venv
source venv/bin/activate

# 安装依赖
pip install -r requirements.txt
\end{lstlisting}

\end{document}
\section{大规模工况定义}

\begin{frame}
    \frametitle{大规模工况定义}
    \textbf{设置}: 长度 10km,时长 4h,采用上游流量、下游水位边界。正式模拟前预热1h。\\模拟过程中前一小时流量 $Q_{in}$ 线性减半。
    \begin{figure}
        \centering
        \includegraphics[width=0.9\textwidth, height=0.75\textheight, keepaspectratio]{images/scenario.png}
    \end{figure}
\end{frame}

\section{阶段二:理想基准}

\begin{frame}
    \frametitle{阶段二:无噪声基准 - 水深 (h)}
    \textbf{目的}: 验证网络容量。预测值 (右) 与真值 (左) 完美重合。
    \begin{figure}
        \centering
        \includegraphics[width=0.9\textwidth, height=0.75\textheight, keepaspectratio]{images/res_clean_h.png}
    \end{figure}
\end{frame}

\begin{frame}
    \frametitle{阶段二:无噪声基准 - 流速 (u)}
    流速预测同样精准,无伪影。
    \begin{figure}
        \centering
        \includegraphics[width=0.9\textwidth, height=0.75\textheight, keepaspectratio]{images/res_clean_u.png}
    \end{figure}
\end{frame}

\begin{frame}
    \frametitle{阶段二:Demo 误差分析}
    上游误差沿特征线向下游传播,上游1h后的稳定边界特征完全丢失
    \begin{figure}
        \centering
        \includegraphics[width=0.9\textwidth, height=0.75\textheight, keepaspectratio]{images/res_clean_err.png}
    \end{figure}
\end{frame}
\section{阶段五:最终方案}

\begin{frame}
    \frametitle{改进一:人工粘性 (Artificial Viscosity)}
    \begin{columns}
        \begin{column}{0.6\textwidth}
            \begin{block}{控制方程修正}
                \Large
                \[ \frac{\partial \mathbf{U}}{\partial t} + \frac{\partial \mathbf{F}}{\partial x} = \mathbf{S} + {\color{red} \nu \frac{\partial^2 \mathbf{U}}{\partial x^2}} \]
            \end{block}

            \vspace{0.5cm}
            \textbf{为何引入?}
            \begin{itemize}
                \item \textbf{物理层}: 模拟流体微团内摩擦,\textbf{抹平激波间断}。
                \item \textbf{优化层}: 平滑 Loss Landscape,防止 \textbf{HMC 采样发散}。
            \end{itemize}
        \end{column}

        \begin{column}{0.4\textwidth}
            \begin{figure}
                \centering
                \includegraphics[width=\textwidth]{images/res_noisy_ts.png}
                \caption{目标:消除此类数值震荡}
            \end{figure}
        \end{column}
    \end{columns}
\end{frame}

\begin{frame}
    \frametitle{改进二:傅里叶特征嵌入 (Fourier Features)}
    \begin{columns}
        \begin{column}{0.6\textwidth}
            \begin{block}{坐标映射 (放大变化)}
                \Large
                \[ \gamma(\mathbf{x}) = \left[ \begin{array}{c} \cos(2\pi \mathbf{B}\mathbf{x}) \\ \sin(2\pi \mathbf{B}\mathbf{x}) \end{array} \right] \]
                \normalsize
                \[ \mathbf{B} \sim \mathcal{N}(0, \sigma^2) \]
            \end{block}

            \textbf{核心机制}
            \begin{itemize}
                \item \textbf{问题}: 神经网络$\tanh$具有低通滤波器的特性,导致神经网络出现了\textbf{谱偏置}问题 ,偏爱低频信息。
                \item \textbf{对策}: 映射至高维频率空间,强迫网络关注高频细节。
            \end{itemize}
        \end{column}

        \begin{column}{0.4\textwidth}
            \begin{figure}
                \centering
                \adjincludegraphics[width=\textwidth, trim={0.45\width} 0 0 0, clip]{images/res_int_h.png}
                \caption{目标:修正过度平滑 (丢失波前)}
            \end{figure}
        \end{column}    \end{columns}
\end{frame}

\begin{frame}
    \frametitle{阶段五结果:精准预测 - 水深 (h)}
    完美重建流场,伪影消失,强噪声鲁棒性。
    \begin{figure}
        \centering
        \includegraphics[width=0.9\textwidth, height=0.75\textheight, keepaspectratio]{images/res_final_h.png}
    \end{figure}
\end{frame}

\begin{frame}
    \frametitle{阶段五结果:精准预测 - 流速 (u)}
    流速场恢复清晰,细节保留完整。
    \begin{figure}
        \centering
        \includegraphics[width=0.9\textwidth, height=0.75\textheight, keepaspectratio]{images/res_final_u.png}
    \end{figure}
\end{frame}

\begin{frame}
    \frametitle{阶段五结果:全域误差分析}
    全域误差显著降低,无明显结构性偏差。
    \begin{figure}
        \centering
        \includegraphics[width=0.9\textwidth, height=0.75\textheight, keepaspectratio]{images/res_final_err.png}
    \end{figure}
\end{frame}

\begin{frame}
    \frametitle{阶段五结果:训练收敛性 (Loss)}
    Loss 曲线平稳下降,无过拟合或震荡迹象。
    \begin{figure}
        \centering
        \includegraphics[width=0.9\textwidth, height=0.75\textheight, keepaspectratio]{images/res_final_loss.png}
    \end{figure}
\end{frame}

\begin{frame}
    \frametitle{阶段五:不确定性量化 (水深 h)}
    \textbf{95\% CI}: 蓝色阴影覆盖良好,不确定性随时间合理扩散。
    \begin{figure}
        \centering
        \includegraphics[width=0.9\textwidth, height=0.75\textheight, keepaspectratio]{images/res_final_ts.png}
    \end{figure}
\end{frame}

\begin{frame}
    \frametitle{阶段五:不确定性量化 (流量 Q)}
    准确捕捉流量减半动态,物理守恒性得到保持。
    \begin{figure}
        \centering
        \includegraphics[width=0.9\textwidth, height=0.75\textheight, keepaspectratio]{images/res_final_ts_q.png}
    \end{figure}
\end{frame}

\begin{frame}
    \frametitle{总结}
    \begin{itemize}
        \item \textbf{当前结论}:BPINN 可以通过人工粘性、傅里叶特征嵌入等技术,有效应对噪声和频谱偏置问题,实现对含噪声简单流场的预测。\\
        \item \textbf{存在问题}: 对于复杂流场(如渐变段、多段渠道联合),当前方法的可行性未经验证。\\
        \item \textbf{存在问题}: 目前的实现还是对流场的复现,也即$f(x,t) = [u,h]$,但我们事实上希望的是一个预测模型,即$f(x,t|BC,IC) = [u,h]$,这需要进一步的研究工作。
    \end{itemize}
    \vspace{1cm}
    \centering
\end{frame}
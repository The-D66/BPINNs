\section{阶段三:噪声挑战}
\begin{frame}
    \frametitle{阶段三:工况设置}
    \textbf{设置}: 在阶段二的基础上,向观测数据添加噪声,模拟实际测量误差对模型的影响。
    \begin{figure}
        \centering
        \includegraphics[width=0.9\textwidth, height=0.8\textheight, keepaspectratio]{images/noisy_solution_full_domain.png}
    \end{figure}
\end{frame}

\begin{frame}
    \frametitle{阶段三:全域水深模拟结果 (h)}
    加入 5\% 高斯噪声后,水深预测出现严重伪影。
    \begin{figure}
        \centering
        \includegraphics[width=0.9\textwidth, height=0.75\textheight, keepaspectratio]{images/res_noisy_h.png}
    \end{figure}
\end{frame}

\begin{frame}
    \frametitle{阶段三:全域流速模拟结果 (u)}
    流速场也出现剧烈震荡,无法使用。
    \begin{figure}
        \centering
        \includegraphics[width=0.9\textwidth, height=0.75\textheight, keepaspectratio]{images/res_noisy_u.png}
    \end{figure}
\end{frame}

\begin{frame}
    \frametitle{阶段三:误差传播机制分析}
    误差连片聚集分布,沿特征线扩散。
    \begin{columns}
        % 左栏
        \begin{column}{0.3\textwidth}
            \begin{enumerate}
                \item 	sol(内部误差) = 0.2
                \item par(先验误差) = 0.0
                \item bnd(边界误差) = 2.0
                \item pde(物理误差) = 0.005
                \item noise\_h\_std\_phys (水深噪声标准差) = 0.2
                \item noise\_Q\_std\_phys (流量噪声标准差) = 0.5
                \item weight\_std (权重标准差) = 0.5
            \end{enumerate}
        \end{column}

        % 右栏
        \begin{column}{0.7\textwidth}
            \begin{figure}
                \centering
                \includegraphics[width=0.9\textwidth, height=0.75\textheight, keepaspectratio]{images/res_noisy_err.png}
            \end{figure}
        \end{column}
    \end{columns}

\end{frame}

\begin{frame}
    \frametitle{阶段三:时序分析 (水深 h)}
    预测值剧烈震荡,物理守恒性丧失。\\黑色:FEM解。红色:带误差的模型训练输入。蓝色:BPINN预测值。
    \begin{figure}
        \centering
        \includegraphics[width=0.9\textwidth, height=0.75\textheight, keepaspectratio]{images/res_noisy_ts.png}
    \end{figure}
\end{frame}

\begin{frame}
    \frametitle{阶段三:时序分析 (流量 Q)}
    预测值剧烈震荡,物理守恒性丧失。
    \begin{figure}
        \centering
        \includegraphics[width=0.9\textwidth, height=0.75\textheight, keepaspectratio]{images/res_noisy_ts_q.png}
    \end{figure}
\end{frame}
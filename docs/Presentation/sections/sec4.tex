\section{阶段四:频谱陷阱}
\begin{frame}
    \frametitle{阶段四:工况设置}
    \textbf{设置}: 仅在内部观测数据添加噪声,模拟模型理论边界+实测误差对模型的影响。
    \begin{figure}
        \centering
        \includegraphics[width=0.9\textwidth, height=0.8\textheight, keepaspectratio]{images/noisy_solution_masked.png}
    \end{figure}
\end{frame}

\begin{frame}
    \frametitle{阶段四:频谱陷阱 - 水深 (h)}
    结果过于平滑,丢失波峰。上游流量边界在(x=0m,t=3600s)时流量的一阶导数不连续,网络未能捕捉到该变化。
    \begin{figure}
        \centering
        \includegraphics[width=0.9\textwidth, height=0.75\textheight, keepaspectratio]{images/res_int_h.png}
    \end{figure}
\end{frame}

\begin{frame}
    \frametitle{阶段四:频谱陷阱 - 流速 (u)}
    流速同样平滑化,未能捕捉锐利变化。
    \begin{figure}
        \centering
        \includegraphics[width=0.9\textwidth, height=0.75\textheight, keepaspectratio]{images/res_int_u.png}
    \end{figure}
\end{frame}

\begin{frame}
    \frametitle{阶段四:系统性偏差分析}
    低频偏差 (Spectral Bias) 导致结构性误差。
    \begin{figure}
        \centering
        \includegraphics[width=0.9\textwidth, height=0.75\textheight, keepaspectratio]{images/res_int_err.png}
    \end{figure}
\end{frame}

\begin{frame}
    \frametitle{阶段四:时序分析 (水深 h)}
    预测均值 (蓝虚线) 明显偏离真值。
    \begin{figure}
        \centering
        \includegraphics[width=0.9\textwidth, height=0.75\textheight, keepaspectratio]{images/res_int_ts.png}
    \end{figure}
\end{frame}

\begin{frame}
    \frametitle{阶段四:时序分析 (流量 Q)}
    波前捕捉失败。
    \begin{figure}
        \centering
        \includegraphics[width=0.9\textwidth, height=0.75\textheight, keepaspectratio]{images/res_int_ts_q.png}
    \end{figure}
\end{frame}
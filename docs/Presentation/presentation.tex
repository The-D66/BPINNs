\documentclass[aspectratio=169]{beamer}
\usepackage[UTF8]{ctex}
\usepackage{graphicx}
\usepackage[export]{adjustbox}
\usepackage{booktabs}
\usepackage{amsmath}

\usetheme{Madrid}
\usecolortheme{beaver}

% Adjust margins to maximize image space
\setbeamersize{text margin left=0.5cm, text margin right=0.5cm}

\title{基于 BPINNs 的圣维南方程求解与不确定性量化}
\author{景象}
\date{\today}

\begin{document}

\frame{\titlepage}

\section{背景知识:物理信息神经网络 (PINN)}

\begin{frame}
    \frametitle{PINN 基本原理}
    \textbf{核心思想}: 将物理定律(偏微分方程 PDE)作为正则化项加入神经网络的损失函数中,实现数据与物理双驱动。
    \vspace{0.5cm}

    \begin{columns}
        \begin{column}{0.5\textwidth}
            \begin{itemize}
                \item \textbf{数据驱动}: 拟合稀疏观测数据(边界、初始条件)。
                \item \textbf{物理驱动}: 最小化 PDE 残差,无需网格。
                \item \textbf{技术关键}: 自动微分 (Automatic Differentiation) 精确计算导数。
            \end{itemize}
        \end{column}
        \begin{column}{0.5\textwidth}
            \begin{block}{通用近似}
                \[ \hat{u}(x,t) \approx \mathcal{NN}(x,t; \theta) \]
            \end{block}
        \end{column}
    \end{columns}
\end{frame}

\begin{frame}
    \frametitle{数学表述}
    \textbf{损失函数} $\mathcal{L}$ 由数据误差 $\mathcal{L}_{data}$ 和物理残差 $\mathcal{L}_{PDE}$ 组成:

    \Large
    \[ \mathcal{L}(\theta) = w_{data}\mathcal{L}_{data} + w_{PDE} \mathcal{L}_{PDE} \]
    \normalsize

    \vspace{0.3cm}
    \begin{itemize}
        \item \textbf{数据损失} (观测值匹配):
              \[ \mathcal{L}_{data} = \frac{1}{N_d} \sum_{i=1}^{N_d} \| \hat{u}(x_i,t_i) - u_{obs}^i \|^2 \]

        \item \textbf{物理残差} (控制方程 $f(u, u_t, u_x, \dots) = 0$):
              \[ \mathcal{L}_{PDE} = \frac{1}{N_f} \sum_{j=1}^{N_f} \| f(\hat{u}(x_j,t_j); \frac{\partial \hat{u}}{\partial t}, \nabla \hat{u}) \|^2 \]
    \end{itemize}
\end{frame}

\begin{frame}
    \frametitle{面临的主要问题}
    \begin{enumerate}
        \item \textbf{优化困难 }
              \begin{itemize}
                  \item 损失函数不同损失项 ($\mathcal{L}_{data}$ vs $\mathcal{L}_{PDE}$) 梯度量级差异大,难以平衡。需要仔细调参才可以较好训练。
              \end{itemize}
              \vspace{0.2cm}
        \item \textbf{谱偏置 }
              \begin{itemize}
                  \item 深度神经网络倾向于优先学习低频函数 (F-Principle)。
                  \item 难以捕捉高频振荡、剧烈变化的波前或多尺度特征。
              \end{itemize}
              \vspace{0.2cm}
        \item \textbf{因果性缺失}
              \begin{itemize}
                  \item 时间演化问题中,标准 PINN 缺乏时间因果性,导致解在整个时空域同时收敛,而非按时间推进。
              \end{itemize}
    \end{enumerate}
\end{frame}

\begin{frame}
    \frametitle{常用改进方法}
    针对上述问题,学术界提出了多种改进方案:
    \begin{table}
        \centering
        \small
        \begin{tabular}{l|l}
            \toprule
            \textbf{改进方向} & \textbf{代表性方法}                              \\
            \midrule
            \textbf{特征工程} & \textbf{傅里叶特征嵌入 (Fourier Features)} (解决谱偏置) \\
                          & 自适应激活函数 (Adaptive Activation Functions)     \\
            \midrule
            \textbf{损失加权} & 自适应权重 (Gradient Normalization, SoftAdapt)   \\
                          & 神经正切核 (Neural Tangent Kernel, NTK) 加权       \\
            \midrule
            \textbf{训练策略} & 课程学习 (Curriculum Learning)                  \\
                          & 因果训练 (Causal Training), 序列到序列学习             \\
            \midrule
            \textbf{模型架构} & 区域分解 (XPINNs), 算子学习 (DeepONet, FNO)         \\
            \bottomrule
        \end{tabular}
    \end{table}
\end{frame}

\begin{frame}
    \frametitle{项目基本情况}

    \begin{enumerate}
        \item \textbf{目标}: 发展基于 BPINNs 的圣维南方程求解与不确定性量化方法,实现利用有限观测数据对中线工程全线水情进行模拟
        \item \textbf{直接采用监测数据训练难度大}: 利用水动力模拟结果开展训练
        \item \textbf{闸门处loss较大}: 仅对渠道进行训练(事后证明渠道模拟只是rmse小,其实渠道才是真正谬误所在)
        \item \textbf{本次汇报内容}: 最简化圣维南方程,仅拟合矩形河道单一工况加噪声的模拟结果,最后较为成功
    \end{enumerate}


\end{frame}


\begin{frame}
    \frametitle{项目演进路线图}
    \begin{enumerate}
        \item \textbf{阶段一:原型验证} - 小规模问题,可行性验证。
        \item \textbf{阶段二:理想基准} - 大规模问题,无噪声验证。
        \item \textbf{阶段三:噪声挑战} - 引入噪声,暴露双曲型方程弱点。
        \item \textbf{阶段四:频谱陷阱} - 稳定边界数据后,揭示低频偏差问题。
        \item \textbf{阶段五:最终方案} - 人工粘性 + 频域特征。
    \end{enumerate}
    \vspace{0.5cm}
    \indent 实际实施的顺序是:3 → 4 → 1 → 5 → 2。
\end{frame}

\section{阶段一:原理验证}

\begin{frame}
    \frametitle{阶段一:Demo 预测结果 - 水深 (h)}
    \textbf{小规模验证}: 矩形渠道平底,长20m,模拟32s,边界条件不变\\
    FDM(左) vs PINN (右)
    \begin{figure}
        \centering
        \includegraphics[width=0.9\textwidth, height=0.75\textheight, keepaspectratio]{images/demo_h.png}
    \end{figure}
\end{frame}

\begin{frame}
    \frametitle{阶段一:Demo 预测结果 - 流速 (u)}
    \textbf{小规模验证}: 流速场捕捉准确。
    \begin{figure}
        \centering
        \includegraphics[width=0.9\textwidth, height=0.75\textheight, keepaspectratio]{images/demo_u.png}
    \end{figure}
\end{frame}

\begin{frame}
    \frametitle{阶段一:Demo 误差分析}
    误差分布均匀且量级极低,水深$h$误差约为5.21e-3\ m,流速$u$误差约为3.53e-2\ m/s。
    \begin{figure}
        \centering
        \includegraphics[width=0.9\textwidth, height=0.75\textheight, keepaspectratio]{images/demo_err.png}
    \end{figure}
\end{frame}


\section{大规模工况定义}

\begin{frame}
    \frametitle{大规模工况定义}
    \textbf{设置}: 长度 10km,时长 4h,采用上游流量、下游水位边界。正式模拟前预热1h。\\模拟过程中前一小时流量 $Q_{in}$ 线性减半。
    \begin{figure}
        \centering
        \includegraphics[width=0.9\textwidth, height=0.75\textheight, keepaspectratio]{images/scenario.png}
    \end{figure}
\end{frame}

\section{阶段二:理想基准}

\begin{frame}
    \frametitle{阶段二:无噪声基准 - 水深 (h)}
    \textbf{目的}: 验证网络容量。预测值 (右) 与真值 (左) 完美重合。
    \begin{figure}
        \centering
        \includegraphics[width=0.9\textwidth, height=0.75\textheight, keepaspectratio]{images/res_clean_h.png}
    \end{figure}
\end{frame}

\begin{frame}
    \frametitle{阶段二:无噪声基准 - 流速 (u)}
    流速预测同样精准,无伪影。
    \begin{figure}
        \centering
        \includegraphics[width=0.9\textwidth, height=0.75\textheight, keepaspectratio]{images/res_clean_u.png}
    \end{figure}
\end{frame}

\begin{frame}
    \frametitle{阶段二:Demo 误差分析}
    上游误差沿特征线向下游传播,上游1h后的稳定边界特征完全丢失
    \begin{figure}
        \centering
        \includegraphics[width=0.9\textwidth, height=0.75\textheight, keepaspectratio]{images/res_clean_err.png}
    \end{figure}
\end{frame}
\section{阶段三:噪声挑战}
\begin{frame}
    \frametitle{阶段三:工况设置}
    \textbf{设置}: 在阶段二的基础上,向观测数据添加噪声,模拟实际测量误差对模型的影响。
    \begin{figure}
        \centering
        \includegraphics[width=0.9\textwidth, height=0.8\textheight, keepaspectratio]{images/noisy_solution_full_domain.png}
    \end{figure}
\end{frame}

\begin{frame}
    \frametitle{阶段三:全域水深模拟结果 (h)}
    加入 5\% 高斯噪声后,水深预测出现严重伪影。
    \begin{figure}
        \centering
        \includegraphics[width=0.9\textwidth, height=0.75\textheight, keepaspectratio]{images/res_noisy_h.png}
    \end{figure}
\end{frame}

\begin{frame}
    \frametitle{阶段三:全域流速模拟结果 (u)}
    流速场也出现剧烈震荡,无法使用。
    \begin{figure}
        \centering
        \includegraphics[width=0.9\textwidth, height=0.75\textheight, keepaspectratio]{images/res_noisy_u.png}
    \end{figure}
\end{frame}

\begin{frame}
    \frametitle{阶段三:误差传播机制分析}
    误差连片聚集分布,沿特征线扩散。
    \begin{columns}
        % 左栏
        \begin{column}{0.3\textwidth}
            \begin{enumerate}
                \item 	sol(内部误差) = 0.2
                \item par(先验误差) = 0.0
                \item bnd(边界误差) = 2.0
                \item pde(物理误差) = 0.005
                \item noise\_h\_std\_phys (水深噪声标准差) = 0.2
                \item noise\_Q\_std\_phys (流量噪声标准差) = 0.5
                \item weight\_std (权重标准差) = 0.5
            \end{enumerate}
        \end{column}

        % 右栏
        \begin{column}{0.7\textwidth}
            \begin{figure}
                \centering
                \includegraphics[width=0.9\textwidth, height=0.75\textheight, keepaspectratio]{images/res_noisy_err.png}
            \end{figure}
        \end{column}
    \end{columns}

\end{frame}

\begin{frame}
    \frametitle{阶段三:时序分析 (水深 h)}
    预测值剧烈震荡,物理守恒性丧失。\\黑色:FEM解。红色:带误差的模型训练输入。蓝色:BPINN预测值。
    \begin{figure}
        \centering
        \includegraphics[width=0.9\textwidth, height=0.75\textheight, keepaspectratio]{images/res_noisy_ts.png}
    \end{figure}
\end{frame}

\begin{frame}
    \frametitle{阶段三:时序分析 (流量 Q)}
    预测值剧烈震荡,物理守恒性丧失。
    \begin{figure}
        \centering
        \includegraphics[width=0.9\textwidth, height=0.75\textheight, keepaspectratio]{images/res_noisy_ts_q.png}
    \end{figure}
\end{frame}
\section{阶段四:频谱陷阱}
\begin{frame}
    \frametitle{阶段四:工况设置}
    \textbf{设置}: 仅在内部观测数据添加噪声,模拟模型理论边界+实测误差对模型的影响。
    \begin{figure}
        \centering
        \includegraphics[width=0.9\textwidth, height=0.8\textheight, keepaspectratio]{images/noisy_solution_masked.png}
    \end{figure}
\end{frame}

\begin{frame}
    \frametitle{阶段四:频谱陷阱 - 水深 (h)}
    结果过于平滑,丢失波峰。上游流量边界在(x=0m,t=3600s)时流量的一阶导数不连续,网络未能捕捉到该变化。
    \begin{figure}
        \centering
        \includegraphics[width=0.9\textwidth, height=0.75\textheight, keepaspectratio]{images/res_int_h.png}
    \end{figure}
\end{frame}

\begin{frame}
    \frametitle{阶段四:频谱陷阱 - 流速 (u)}
    流速同样平滑化,未能捕捉锐利变化。
    \begin{figure}
        \centering
        \includegraphics[width=0.9\textwidth, height=0.75\textheight, keepaspectratio]{images/res_int_u.png}
    \end{figure}
\end{frame}

\begin{frame}
    \frametitle{阶段四:系统性偏差分析}
    低频偏差 (Spectral Bias) 导致结构性误差。
    \begin{figure}
        \centering
        \includegraphics[width=0.9\textwidth, height=0.75\textheight, keepaspectratio]{images/res_int_err.png}
    \end{figure}
\end{frame}

\begin{frame}
    \frametitle{阶段四:时序分析 (水深 h)}
    预测均值 (蓝虚线) 明显偏离真值。
    \begin{figure}
        \centering
        \includegraphics[width=0.9\textwidth, height=0.75\textheight, keepaspectratio]{images/res_int_ts.png}
    \end{figure}
\end{frame}

\begin{frame}
    \frametitle{阶段四:时序分析 (流量 Q)}
    波前捕捉失败。
    \begin{figure}
        \centering
        \includegraphics[width=0.9\textwidth, height=0.75\textheight, keepaspectratio]{images/res_int_ts_q.png}
    \end{figure}
\end{frame}
\section{阶段五:最终方案}

\begin{frame}
    \frametitle{改进一:人工粘性 (Artificial Viscosity)}
    \begin{columns}
        \begin{column}{0.6\textwidth}
            \begin{block}{控制方程修正}
                \Large
                \[ \frac{\partial \mathbf{U}}{\partial t} + \frac{\partial \mathbf{F}}{\partial x} = \mathbf{S} + {\color{red} \nu \frac{\partial^2 \mathbf{U}}{\partial x^2}} \]
            \end{block}

            \vspace{0.5cm}
            \textbf{为何引入?}
            \begin{itemize}
                \item \textbf{物理层}: 模拟流体微团内摩擦,\textbf{抹平激波间断}。
                \item \textbf{优化层}: 平滑 Loss Landscape,防止 \textbf{HMC 采样发散}。
            \end{itemize}
        \end{column}

        \begin{column}{0.4\textwidth}
            \begin{figure}
                \centering
                \includegraphics[width=\textwidth]{images/res_noisy_ts.png}
                \caption{目标:消除此类数值震荡}
            \end{figure}
        \end{column}
    \end{columns}
\end{frame}

\begin{frame}
    \frametitle{改进二:傅里叶特征嵌入 (Fourier Features)}
    \begin{columns}
        \begin{column}{0.6\textwidth}
            \begin{block}{坐标映射 (放大变化)}
                \Large
                \[ \gamma(\mathbf{x}) = \left[ \begin{array}{c} \cos(2\pi \mathbf{B}\mathbf{x}) \\ \sin(2\pi \mathbf{B}\mathbf{x}) \end{array} \right] \]
                \normalsize
                \[ \mathbf{B} \sim \mathcal{N}(0, \sigma^2) \]
            \end{block}

            \textbf{核心机制}
            \begin{itemize}
                \item \textbf{问题}: 神经网络$\tanh$具有低通滤波器的特性,导致神经网络出现了\textbf{谱偏置}问题 ,偏爱低频信息。
                \item \textbf{对策}: 映射至高维频率空间,强迫网络关注高频细节。
            \end{itemize}
        \end{column}

        \begin{column}{0.4\textwidth}
            \begin{figure}
                \centering
                \adjincludegraphics[width=\textwidth, trim={0.45\width} 0 0 0, clip]{images/res_int_h.png}
                \caption{目标:修正过度平滑 (丢失波前)}
            \end{figure}
        \end{column}    \end{columns}
\end{frame}

\begin{frame}
    \frametitle{阶段五结果:精准预测 - 水深 (h)}
    完美重建流场,伪影消失,强噪声鲁棒性。
    \begin{figure}
        \centering
        \includegraphics[width=0.9\textwidth, height=0.75\textheight, keepaspectratio]{images/res_final_h.png}
    \end{figure}
\end{frame}

\begin{frame}
    \frametitle{阶段五结果:精准预测 - 流速 (u)}
    流速场恢复清晰,细节保留完整。
    \begin{figure}
        \centering
        \includegraphics[width=0.9\textwidth, height=0.75\textheight, keepaspectratio]{images/res_final_u.png}
    \end{figure}
\end{frame}

\begin{frame}
    \frametitle{阶段五结果:全域误差分析}
    全域误差显著降低,无明显结构性偏差。
    \begin{figure}
        \centering
        \includegraphics[width=0.9\textwidth, height=0.75\textheight, keepaspectratio]{images/res_final_err.png}
    \end{figure}
\end{frame}

\begin{frame}
    \frametitle{阶段五结果:训练收敛性 (Loss)}
    Loss 曲线平稳下降,无过拟合或震荡迹象。
    \begin{figure}
        \centering
        \includegraphics[width=0.9\textwidth, height=0.75\textheight, keepaspectratio]{images/res_final_loss.png}
    \end{figure}
\end{frame}

\begin{frame}
    \frametitle{阶段五:不确定性量化 (水深 h)}
    \textbf{95\% CI}: 蓝色阴影覆盖良好,不确定性随时间合理扩散。
    \begin{figure}
        \centering
        \includegraphics[width=0.9\textwidth, height=0.75\textheight, keepaspectratio]{images/res_final_ts.png}
    \end{figure}
\end{frame}

\begin{frame}
    \frametitle{阶段五:不确定性量化 (流量 Q)}
    准确捕捉流量减半动态,物理守恒性得到保持。
    \begin{figure}
        \centering
        \includegraphics[width=0.9\textwidth, height=0.75\textheight, keepaspectratio]{images/res_final_ts_q.png}
    \end{figure}
\end{frame}

\begin{frame}
    \frametitle{总结}
    \begin{itemize}
        \item \textbf{当前结论}:BPINN 可以通过人工粘性、傅里叶特征嵌入等技术,有效应对噪声和频谱偏置问题,实现对含噪声简单流场的预测。\\
        \item \textbf{存在问题}: 对于复杂流场(如渐变段、多段渠道联合),当前方法的可行性未经验证。\\
        \item \textbf{存在问题}: 目前的实现还是对流场的复现,也即$f(x,t) = [u,h]$,但我们事实上希望的是一个预测模型,即$f(x,t|BC,IC) = [u,h]$,这需要进一步的研究工作。
    \end{itemize}
    \vspace{1cm}
    \centering
\end{frame}
\section{特殊工况:仅边界数据}

\begin{frame}
\frametitle{特殊工况:无域内数据 (Boundary Data Only)}
\begin{columns}
    \begin{column}{0.6\textwidth}
        \begin{block}{实验设置}
        \begin{itemize}
            \item \textbf{数据}: 仅提供初始条件 (IC, $t=0$) 和边界条件 (BC, $x=0, L$) 的观测值。
            \item \textbf{约束}: 域内仅通过 PDE 残差 ($Res_{PDE} \approx 0$) 进行物理约束。
        \end{itemize}
        \end{block}

        \textbf{结果观测 - 水深 (h)}
        \begin{itemize}
            \item 预测完全失败。
            \item 尽管边界条件被强制满足,但波形无法向域内传播。
            \item 网络坍缩为一个简单的平面解,无法捕捉行波动态。
        \end{itemize}
    \end{column}

    \begin{column}{0.4\textwidth}
        \begin{figure}
            \centering
            \includegraphics[width=\textwidth]{images/res_no_internal_h.png}
            \caption{水深预测 (仅边界数据)}
        \end{figure}
    \end{column}
\end{columns}
\end{frame}

\begin{frame}
\frametitle{仅边界数据:时序分析}
\begin{columns}
    \begin{column}{0.5\textwidth}
        \begin{figure}
            \centering
            \includegraphics[width=\textwidth]{images/res_no_internal_ts_h.png}
            \caption{水深 (h) 时序预测}
        \end{figure}
    \end{column}
    \begin{column}{0.5\textwidth}
        \begin{figure}
            \centering
            \includegraphics[width=\textwidth]{images/res_no_internal_ts_q.png}
            \caption{流量 (Q) 时序预测}
        \end{figure}
    \end{column}
\end{columns}
\begin{block}{分析}
模型完全未能捕捉到随时间演化的动态行为,水深和流量的时序曲线都呈现出静态或极度平滑的预测,与真实物理过程严重不符。这进一步印证了仅靠边界数据难以有效求解双曲型 PDE。
\end{block}
\end{frame}

\begin{frame}
\frametitle{失败原因分析}
\begin{columns}
    \begin{column}{0.5\textwidth}
        \begin{figure}
            \centering
            \includegraphics[width=\textwidth]{images/res_no_internal_err.png}
            \caption{全域误差分布}
        \end{figure}
    \end{column}

    \begin{column}{0.5\textwidth}
        \begin{block}{为何 PINN 无法仅靠边界求解双曲方程?}
        \begin{itemize}
            \item \textbf{缺乏因果性}: PINN 的 Loss 是空间点上的平均求和。优化器“看不见”时间的先后顺序,无法沿着特征线将边界信息传递到内部。
            \item \textbf{传播困难}: 对于长时间积分问题,边界信息需要穿越整个时空域。梯度在反向传播中容易消失或被局部极小值阻断。
            \item \textbf{结论}: 对于圣维南这类双曲系统,仅靠 PDE 约束很难收敛,\textbf{稀疏的内部观测数据是重要的}。
        \end{itemize}
        \end{block}
    \end{column}
\end{columns}
\end{frame}

\section{改进方法:算符学习}

\begin{frame}
\frametitle{算符学习结果}
    \begin{figure}
        \centering
        \includegraphics[width=0.95\textwidth, height=0.85\textheight, keepaspectratio]{images/results_visualization_improved.png}
    \end{figure}
\end{frame}

\end{document}

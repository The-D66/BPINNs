\documentclass[aspectratio=169]{beamer}

% 中文支持
\usepackage{xeCJK}
\setCJKmainfont[BoldFont=PingFang SC Semibold, ItalicFont=PingFang SC Light]{PingFang SC}
\setsansfont[BoldFont=PingFang SC Semibold]{PingFang SC} % Beamer 默认使用无衬线字体

% 宏包
\usepackage{amsmath}
\usepackage{amssymb}
\usepackage{graphicx}
\usepackage{tikz}
\usetikzlibrary{shapes, arrows, positioning, fit, calc}
\usepackage{booktabs}
\usepackage{multicol}
\usepackage{hyperref}

% 主题设置
\usetheme{Madrid}
\usecolortheme{default}
\setbeamercovered{transparent}

% 自定义颜色
\definecolor{DeepBlue}{RGB}{0,51,102}
\setbeamercolor{structure}{fg=DeepBlue}
\setbeamercolor{title}{bg=DeepBlue, fg=white}
\setbeamercolor{frametitle}{bg=DeepBlue, fg=white}

% 标题信息
\title[PINN 综述]{PDE 问题的物理信息神经网络:综述 \\ Physics-informed neural networks for PDE problems: a comprehensive review}
\author{汇报人:Gemini Agent}
\institute{基于 Luo 等 (2025) 的论文}
\date{\today}

\begin{document}

% 1. 标题页
\begin{frame}
    \titlepage
\end{frame}

% 2. 目录
\begin{frame}{目录}
    \tableofcontents
\end{frame}

% ============================================================ 
% Section 1: 背景与基础
% ============================================================ 
\section{背景与数理基础}

\begin{frame}{科学计算的新范式:AI for Science}
    \begin{columns}
        \column{0.5\textwidth}
        \begin{block}{传统数值方法 (FEM/FVM)}
            \begin{itemize}
                \item \textbf{优点}:理论成熟,精度高。
                \item \textbf{痛点}:网格生成复杂,高维问题“维数灾难”,反问题求解困难。
            \end{itemize}
        \end{block}
        
        \begin{block}{物理信息机器学习 (PIML)}
            \begin{itemize}
                \item \textbf{核心}:数据 + 物理定律 (PDEs)。
                \item \textbf{优势}:无网格 (Mesh-free),易于处理反问题,融合观测数据。
            \end{itemize}
        \end{block}

                \column{0.5\textwidth}
                \centering
                \includegraphics[height=0.75\textheight, keepaspectratio]{images/Fig_1.png}
                \\ \scriptsize{图 1:物理信息机器学习在各领域的应用}
            \end{columns}
        \end{frame}
        
        \begin{frame}{PINN 的核心数理逻辑}
                \begin{columns}
                    \column{0.55\textwidth}
                    \small
                    \textbf{问题}:求解 $u(x,t)$ 使得
                    $$ f(x, t, \partial_x u, \partial_t u, \lambda) = 0 $$
                    
                    \textbf{损失函数} $\mathcal{L} = \mathcal{L}_{physics} + \mathcal{L}_{data}$
                    
                    \begin{itemize}
                        \item \textbf{物理损失} (残差):
                        $ \mathcal{L}_f = \frac{1}{N_f} \sum |f(x_f, t, \lambda)|^2 $
                        \item \textbf{边界/初始条件}:
                        $ \mathcal{L}_{b/i} = \frac{1}{N_{b/i}} \sum |\tilde{u} - g|^2 $
                        \item \textbf{数据损失} (可选):
                        $ \mathcal{L}_{data} = \frac{1}{N_d} \sum |\tilde{u} - u_{obs}|^2 $
                    \end{itemize}        
                \column{0.45\textwidth}
                \centering
                \includegraphics[width=0.9\textwidth, height=0.65\textheight, keepaspectratio]{images/Fig_2.png}
                \scriptsize{图 2:PINN 框架}
            \end{columns}
        \end{frame}
        
        % ============================================================
        % Section 2: 架构演进
        % ============================================================
        \section{网络架构演进}
        
        \begin{frame}{PINN 架构概览}
            \centering
            \includegraphics[width=\textwidth, height=0.8\textheight, keepaspectratio]{images/Fig_3.png}
            \\
            \vspace{0.5em}
            从简单的 MLP 到 结合 KAN、Transformer 的复杂架构
        \end{frame}
        
        \begin{frame}{主流架构对比 (1)}
            \begin{columns}
                \column{0.5\textwidth}
                \begin{block}{MLP (多层感知机)}
                    \begin{itemize}
                        \item 基础架构,通用近似定理。
                        \item \textbf{局限}:频谱偏差,难捉高频。
                    \end{itemize}
                \end{block}
                \centering
                \includegraphics[height=0.45\textheight, keepaspectratio]{images/Fig_4.png}
        
                \column{0.5\textwidth}
                \begin{block}{CNN (卷积神经网络)}
                    \begin{itemize}
                        \item \textbf{适用}:规则网格,图像类数据。
                        \item \textbf{变体}:PhyGeoNet, f-PICNN。
                    \end{itemize}
                \end{block}
                \centering
                \includegraphics[height=0.45\textheight, keepaspectratio]{images/Fig_5.png}
                \\ \scriptsize{图 5:f-PICNN 架构}
            \end{columns}
        \end{frame}
        
        \begin{frame}{主流架构对比 (2)}
            \begin{columns}
                \column{0.5\textwidth}
                \begin{block}{RNN (循环神经网络)}
                    \begin{itemize}
                        \item \textbf{适用}:时间依赖问题,序列建模。
                        \item \textbf{PhyCRNet}:CNN + ConvLSTM。
                    \end{itemize}
                \end{block}
                \centering
                \includegraphics[height=0.45\textheight, keepaspectratio]{images/Fig_6.png}
        
                        \column{0.48\textwidth}
                        \begin{block}{Transformer}
                            \begin{itemize}
                                \item \textbf{优势}:Attention 捕捉长程依赖。
                                \item \textbf{PINNsFormer}:Wavelet + Attention。
                            \end{itemize}
                        \end{block}
                        \centering
                        \includegraphics[width=\textwidth, height=0.4\textheight, keepaspectratio]{images/Fig_7.png}            \end{columns}
        \end{frame}
        
        \begin{frame}{前沿架构:KAN 与 区域分解}
            \begin{columns}
                \column{0.5\textwidth}
                \begin{block}{Kolmogorov-Arnold Networks (KAN)}
                    \begin{itemize}
                        \item 激活函数在\textbf{边}上。
                        \item \textbf{公式}:$f(x) = \sum \Phi_q (\sum \phi_{q,p}(x_p))$
                        \item \textbf{优势}:高维函数更高效,可解释性。
                    \end{itemize}
                \end{block}
        
                \column{0.5\textwidth}
                \begin{block}{区域分解 (Domain Decomposition)}
                    \begin{itemize}
                        \item \textbf{XPINNs / cPINNs}
                        \item 将复杂域划分为子域,并行求解。
                        \item 通过界面条件拼接。
                    \end{itemize}
                \end{block}
                \centering
                \includegraphics[height=0.4\textheight, keepaspectratio]{images/Fig_9.png}
            \end{columns}
        \end{frame}
% ============================================================ 
% Section 3: 优化方法
% ============================================================ 
\section{关键优化方法}

\begin{frame}{痛点与解决方案概览}
    \centering
    \begin{tikzpicture}[
        node distance=1.5cm,
        block/.style={rectangle, draw=DeepBlue, fill=blue!10, rounded corners, text width=3cm, align=center, minimum height=1.2cm},
        solution/.style={rectangle, draw=red!80!black, fill=red!10, rounded corners, text width=3cm, align=center, minimum height=1.2cm},
        arrow/.style={->, >=stealth, thick, color=DeepBlue}
    ]
        \node (p1) [block] {收敛慢 / 陷入局部极小};
        \node (s1) [solution, right=of p1] {自适应重采样 \\ (RAR, GAS, HA)};
        \node (p2) [block, below=of p1] {多目标损失不平衡 \\ (梯度病态)};
        \node (s2) [solution, right=of p2] {损失重加权 \\ (Loss Reweighting, NTK)};
        \node (p3) [block, below=of p2] {高频/多尺度问题 \\ (频谱偏差)};
        \node (s3) [solution, right=of p3] {特征嵌入 (Fourier) \\ + 维度增强 (DaPINN)};

        \draw [arrow] (p1) -- (s1);
        \draw [arrow] (p2) -- (s2);
        \draw [arrow] (p3) -- (s3);
    \end{tikzpicture}
\end{frame}

\begin{frame}{自适应采样 (Adaptive Resampling)}
    \begin{columns}
        \column{0.6\textwidth}
        \begin{itemize}
            \item \textbf{核心思想}:在 PDE 残差大的区域增加采样点。
            \item \textbf{方法}:
                \begin{itemize}
                    \item \textbf{RAR}:基于残差添加点。
                    \item \textbf{GAS}:高斯混合分布采样。
                    \item \textbf{HA (Hybrid Adaptive)}:结合随机性与残差导向。
                \end{itemize}
            \item \textbf{效果}:显著降低 L2 误差 (见图 10)。
        \end{itemize}

        \column{0.4\textwidth}
        \centering
        \includegraphics[width=\textwidth, height=0.75\textheight, keepaspectratio]{images/Fig_10.png}
        \scriptsize{图 10:HA 采样方法在泊松方程上的性能提升}
    \end{columns}
\end{frame}

\begin{frame}{特征嵌入与维度增强}
    \begin{columns}
        \column{0.5\textwidth}
        \begin{block}{特征嵌入 (Feature Embedding)}
            \begin{itemize}
                \item 解决 MLP 的频谱偏差。
                \item \textbf{傅里叶特征}:映射输入 $x \to [\sin(2\pi B x), \cos(2\pi B x)]$。
            \end{itemize}
        \end{block}
        \begin{block}{维度增强 (DaPINN)}
            \begin{itemize}
                \item 人为扩展网络输入维度。
                \item 提取更丰富特征,提高精度。
            \end{itemize}
        \end{block}

        \column{0.5\textwidth}
        \centering
        \includegraphics[width=0.9\textwidth, height=0.8\textheight, keepaspectratio]{images/Fig_11.png}
        \scriptsize{图 11:DaPINN 框架示意图}
    \end{columns}
\end{frame}

\begin{frame}{DaPINN 实验效果}
    \begin{columns}
        \column{0.5\textwidth}
        \centering
        \includegraphics[width=0.9\textwidth, height=0.65\textheight, keepaspectratio]{images/Fig_12.png}
        \\ \scriptsize{图 12:1D 泊松方程 (DaPINN vs PINN)}
        
        \column{0.5\textwidth}
        \centering
        \includegraphics[width=0.9\textwidth, height=0.65\textheight, keepaspectratio]{images/Fig_13.png}
        \\ \scriptsize{图 13:Burgers 方程 (DaPINN vs PINN)}
    \end{columns}
\end{frame}

% ============================================================ 
% Section 4: 应用与挑战
% ============================================================ 
\section{应用与未来}

\begin{frame}{典型应用场景}
    \begin{itemize}
        \item \textbf{流体动力学}
            \begin{itemize}
                \item 求解 Navier-Stokes 方程。
                \item 根据稀疏观测数据重建流场 (如 PIV 实验)。
                \item 无需完整边界条件即可求解反问题。
            \end{itemize}
        \item \textbf{固体力学}
            \begin{itemize}
                \item 弹性、塑性、断裂力学模拟。
                \item 材料参数反演。
            \end{itemize}
        \item \textbf{光学与电磁学}
            \begin{itemize}
                \item 求解 Maxwell 方程组。
                \item 超材料设计与逆散射问题。
            \end{itemize}
    \end{itemize}
\end{frame}

\begin{frame}{挑战与未来展望}
    \begin{columns}
        \column{0.5\textwidth}
        \begin{block}{当前挑战}
            \begin{enumerate}
                \item \textbf{优化困难}:非凸优化,梯度不平衡。
                \item \textbf{计算效率}:训练时间长,收敛精度瓶颈 ($10^{-3} \sim 10^{-4}$)。
                \item \textbf{高频/多尺度}:传统网络难以捕捉。
            \end{enumerate}
        \end{block}

        \column{0.5\textwidth}
        \begin{block}{未来方向:算子学习 (Operator Learning)}
            \begin{itemize}
                \item 从学习“函数”转向学习“算子”。
                \item \textbf{DeepONet / FNO}。
                \item 一次训练,求解一类方程(不同初始条件/参数)。
                \item 泛化能力更强,推理速度更快。
            \end{itemize}
        \end{block}
    \end{columns}
\end{frame}

\begin{frame}{总结}
    \centering
    \begin{itemize}
        \Large
        \item \textbf{PINN 是连接 数据驱动 与 物理定律 的桥梁。}
        \item \textbf{架构创新} (KAN, Transformer) 不断突破精度极限。
        \item \textbf{优化策略} (自适应采样, 损失平衡) 解决了训练难题。
        \item \textbf{算子学习} 代表了下一代科学计算的发展方向。
    \end{itemize}
    
    \vspace{1cm}
    \textbf{\Large 谢谢! \\ Q \& A}
\end{frame}

\end{document}

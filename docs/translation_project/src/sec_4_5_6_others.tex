\section{软件}

已经开发了许多开源工具和库来促进 PINN 的实施。这些工具抽象了构建 PINN 所涉及的大部分复杂性,使研究人员和从业者更容易将此方法应用于科学和工程中的各种问题。本节概述了一些广泛使用的、可用于实施 PINN 的公共框架和库(表 4)。

\subsection{DeepXDE}

DeepXDE (Lu 等 2021b) 是一个专门设计用于使用 PINN 求解微分方程的开源 Python 库。它通过利用深度学习技术,为定义、训练和求解 ODE 和 PDE 提供了一个高级、易于使用的接口。该库与 TensorFlow 或 PyTorch 无缝集成,允许用户轻松地将神经网络应用于复杂的、现实世界的基于物理的问题。

\subsection{IDRLnet}

IDRLnet (Peng 等 2021) 是一个用于通过 PINN 系统地建模和求解问题的 Python 工具箱。它提供了一种结构化的方式来在 Python 中集成几何对象、数据源、人工神经网络、损失指标和优化器。

\subsection{NeuroDiffEq}

NeuroDiffEq (Chen 等 2020b; Liu 等 2025) 是一个开发的用于使用神经网络求解微分方程的开源 Python 库,特别是在 PyTorch 框架内。它利用人工神经网络的能力来近似常微分方程 (ODE) 和偏微分方程 (PDE) 的解,并服从指定的初始或边界条件。该库专为灵活性而设计,使研究人员和从业者能够解决各种用户定义的问题。此外,它促进了不仅连续而且可微的解的计算,与有限差分和有限元方法等传统数值方法相比具有显著优势。

\subsection{SciANN}

SciANN (Haghighat 和 Juanes 2021a) 是专门为科学计算设计的高级神经网络 API。它建立在 Keras 和 TensorFlow 之上,使科学家能够轻松利用深度学习进行物理信息建模。SciANN 允许研究人员定义输入、输出和约束,从而能够将科学问题制定为深度学习任务。该框架对于求解 PDE 和模拟系统动力学特别有用,使其成为希望将深度学习技术应用于其研究的科学家和工程师的宝贵工具。

\subsection{TensorDiffEq}

TensorDiffEq (McClenny 等 2021) 是构建在 TensorFlow 之上的 Python 包,旨在为 PINN 提供可扩展且高效的求解器。它特别专注于 PINN 的可扩展求解,用于推理和反问题发现。TensorDiffEq 的独特之处在于它完全支持自适应 PINN 求解器,并且是唯一完全开源的多 GPU PINN 解决方案套件。

\begin{table}[htbp]
\centering
\caption{一些用于 PINN 的开源框架}
\label{tab:frameworks}
\begin{tabular}{ll}
\hline
框架名称 & 后端 \\
\hline
DeepXDE & TensorFlow, PyTorch, JAX \\
IDRLnet & PyTorch \\
NeuroDiffEq & PyTorch \\
SciANN & TensorFlow \\
TensorDiffEq & Tensorflow \\
\hline
\end{tabular}
\end{table}

\section{挑战与展望}

PINN 已被证明是跨各种科学和工程学科的多功能工具,为求解传统方法可能难以应对的复杂 PDE 提供了一种鲁棒且灵活的方法。尽管取得了成功,但这方面的研究仍在继续解决理论问题和实际局限性,旨在提高 PINN 的可扩展性、可解释性和可靠性。在下一节中,我们将重点介绍与 PINN 相关的关键挑战,并提出未来研究的潜在方向。

\subsection{PINN 面临的挑战}

PINN 解决流体动力学、材料科学和生物力学等各个领域问题的能力证明了该方法弥合数据驱动模型与传统数值模拟之间差距的潜力。然而,尽管具有令人印象深刻的能力,PINN 仍然面临一些限制和挑战,阻碍了它们的广泛采用。PINN 由三个关键要素组成:(1) 近似底层函数的神经网络,(2) 整合物理定律的物理信息损失函数,以及 (3) 用于训练模型的优化方法。这些要素使 PINN 能够有效地将物理约束嵌入到学习过程中。然而,在实践中管理这些组件的复杂性通常会导致诸如收敛缓慢、对超参数敏感以及难以扩展到高维或多物理场问题等问题。

为了提高 PINN 目前约为 $10^{-3}$ 的收敛精度,需要进行严格的分析来评估 PINN 的学习能力,重点关注泛化、鲁棒性以及物理系统复杂性所施加的限制等方面。一些研究提供了对 PINN 的理论见解 (Krishnapriyan 等 2021; Wang 等 2022a; Hu 等 2022),但仍需对该主题进行更深入的研究。例如,Krishnapriyan 等 (2021) 确定了一个关键的“梯度病态”问题,该问题由于数值刚性而出现,导致反向传播期间的梯度不平衡。这种不平衡阻碍了网络准确满足 PDE 约束和边界条件的能力,通常导致次优解。

PINN 的关键问题之一是难以捕捉快速变化或高频解。传统的优化算法,如 Adam 或 L-BFGS,可能难以在这些场景中有效收敛,导致性能不佳或训练缓慢。此外,不同物理过程相互作用的多物理场问题引入了额外的复杂性,当前的优化技术可能无法很好地处理。为了解决这些挑战,开发为 PINN 量身定制的新优化算法至关重要。这些算法应通过适应问题的特定性质来增强训练过程的鲁棒性和效率。潜在的方法包括设计更复杂的损失函数、结合分层模型或利用高级优化策略。这些进步对于提高 PINN 的可扩展性、准确性和可靠性至关重要,使其更适用于涉及高频动力学和多物理场相互作用的现实世界问题。

另一个挑战是当前的 PINN 无法有效处理嘈杂或不完整的数据。虽然 PINN 将物理定律整合到学习过程中,但它们仍然容易受到训练数据中错误和不一致的影响,这可能会影响预测的准确性。此外,对于多尺度和多物理场问题,PINN 的可扩展性仍然是一个挑战,因为不同物理模型的相互作用通常会导致模型复杂性增加。

此外,仍然需要开发有价值的实验数据集以及相应的物理模型和参数。高质量的数据集对于训练鲁棒且准确的 PINN 模型至关重要,但许多科学和工程领域缺乏可访问的、详细的实验数据,这限制了现实世界应用的潜力。此外,对可集成到 PINN 中的准确且计算高效的物理模型和参数的需求构成了另一个重大障碍。

\subsection{算子学习}

虽然 PINN 通过将物理定律纳入学习过程在解决复杂物理问题方面取得了显著成功,但仍然存在一些挑战阻碍了它们的全部潜力。PINN 的一个主要局限性是它们依赖于物理约束的手动公式化,这对于具有复杂或未知物理机制的问题来说可能变得繁琐且具有挑战性。此外,PINN 的性能对神经网络架构的选择、训练策略以及边界条件的处理高度敏感,这通常需要大量的调整。

近年来,已经向深度算子学习转变,这是一个有前途的范式,通过专注于直接学习控制物理系统的底层算子来扩展 PINN。与依赖于使用神经网络求解特定 PDE 或系统动力学的传统 PINN 不同,算子学习 (Lu 等 2021a; Wang 等 2021b) 旨在学习输入和输出函数(例如 PDE 的解)之间的映射,从而实现更具泛化能力的模型,无需显式的基于物理的约束即可应用于广泛的物理问题。从 PINN 到深度算子学习的转变为未来的研究提供了几个令人兴奋的途径。首先,开发能够有效学习高维算子映射的更鲁棒的训练算法至关重要。这些模型应设计用于处理现实世界应用中经常遇到的多尺度、多物理场问题的复杂性。此外,需要在算子学习中进一步研究数据驱动模型和基于物理模型的集成,特别是在可用数据稀疏或嘈杂的情况下。未来的研究还应探索算子学习模型的可解释性和可解释性,以确保它们在安全关键型工程领域中的部署。

总之,虽然 PINN 为物理信息学习奠定了基础,但向深度算子学习的转变为将该框架推广到更广泛的问题类别提供了令人兴奋的新可能性。在这个方向上的持续研究对于释放神经网络在科学计算和工程应用中的全部潜力至关重要。

\section{结论}

总之,PINN 已成为求解 PDE 的有力工具,通过将物理定律直接嵌入神经网络的损失函数中,从而弥合了传统数值方法与现代机器学习技术之间的差距。

这篇综述论文全面考察了 PINN 在广泛的科学和工程领域的进步和应用。强调了来自不同视角的各种创新 PINN,包括其架构、自适应重采样、损失和激活函数、特征嵌入和大型模型。尽管 PINN 取得了成功,但这篇综述也承认了正在面临的挑战,例如需要更大的批量大小、复杂的训练程序以及处理高维和非线性问题。未来的研究旨在解决这些问题,重点是增强 PINN 的可扩展性、可解释性和可靠性。

PINN 代表了机器学习和计算科学交叉领域的变革性技术。通过将物理知识直接嵌入机器学习模型,它们为求解跨多个学科的复杂 PDE 提供了强大的工具集。随着研究的进展,预计 PINN 将继续发展,解决现有的局限性并将其适用性扩展到新领域,从而为科学发现和工程实践的进步做出贡献。

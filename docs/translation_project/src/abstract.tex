\textbf{摘要}

随着“科学人工智能”(AI for Science)的持续发展,物理信息神经网络(PINN)已成为科学计算和深度学习领域的一种变革性方法,为求解偏微分方程(PDE)和其他复杂物理系统提供了一个鲁棒且灵活的框架。通过将物理定律直接嵌入神经网络的架构中,PINN 能够整合领域特定的知识,确保模型在拟合现有数据的同时遵守已知的物理规律。在本文中,我们全面综述了 PINN 在广泛的 PDE 问题中的最新进展和应用。特别关注了 PINN 的架构、PINN 的数据重采样方法、损失函数和激活函数、特征嵌入方法等。此外,还讨论了 PINN 的潜在未来方向和预期演变。我们旨在为 PDE 问题的 PINN 提供有价值的见解,希望能鼓励这一充满希望的领域的进一步探索和研究。

\textbf{关键词}:科学人工智能 $\cdot$ 物理信息神经网络 $\cdot$ 深度学习 $\cdot$ 偏微分方程(PDE)


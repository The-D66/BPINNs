\section{应用}

该领域的最新进展证明了 PINN 在广泛应用中的有效性。它标志着机器学习与计算物理学交叉领域的重大飞跃。下面,我们将讨论 PINN 正在产生影响的几个关键领域。

\subsection{流体动力学}

PINN 已成为流体力学中的一项关键技术,为求解控制流体运动的 Navier–Stokes 方程 (NSE) 提供了一种新颖的方法 (Cai 等 2021; Mao 等 2020)。物理学与数据的结合对于解决复杂的高维流体流动问题特别强大,在这些问题中,传统的计算流体力学 (CFD) 方法可能难以应对包含噪声数据或有效处理反问题的要求。

PINN 已成功应用于模拟钝体(如圆柱体)后面的三维尾流,它们可以从有限的观测数据重建完整的速度场和压力场。这种能力对于只有部分测量值可用的复杂流动的诊断目的特别有价值,例如在平面粒子图像测速 (PIV) 实验中。此外,PINN 在模拟高速可压缩流动(例如涉及激波的流动)方面发挥了重要作用。在这些场景中,传统的 CFD 方法通常依赖于精确的边界条件,而在实验设置中可能无法轻易获得这些条件。然而,PINN 可以利用密度梯度和表面压力测量值等可用数据来推断完整的流场。即使在缺乏完整的边界条件信息的情况下,这种方法也被证明是有效的,突出了 PINN 在解决不适定问题方面的灵活性。PINN 的应用也扩展到了生物医学流体力学。例如,在血流动力学 (Kissas 等 2020) 中,PINN 已被用于模拟血流动力学、估计壁面剪切应力并预测特定患者动脉网络中的脉搏波传播。

\subsection{固体力学}

在固体力学中,PINN 已被有效地应用于模拟线性弹性、弹塑性、超弹性和断裂力学。PINN 可以处理正向问题,预测给定输入的材料响应。例如,Rao 等 (2021) 使用 PINN 强形式算法求解了线性弹性动力学问题。在解决高度非均匀的正向问题时,与有限差分法 (FDM) 和有限元法 (FEM) 相比,PINN 表现出卓越的精度和效率 (Guo 等 2022)。固体力学的反问题也可以得到解决,主要包括本构方程 (Haghighat 等 2021b) 和几何拓扑 (Zhang 等 2022a)。

PINN 在固体力学中的潜力是巨大的,有望增强我们要对材料行为的理解和预测,从而在结构设计和材料科学中带来创新的解决方案。

\subsection{电磁学和光学}

PINN 已扩展到求解光学和电磁学中的问题,例如 3-D 亥姆霍兹方程和拟线性 PDE 算子。它们已被证明对这些领域的反问题有效,由于电磁场的复杂性质,这些问题可能特别具有挑战性 (Fang 和 Zhan 2020)。Chen 等 (2020a) 利用 PINN 解决光子超材料和纳米光学技术中的逆散射问题。无网格 PINN 已被应用于检索涉及许多相互作用的纳米结构以及多组分纳米粒子的许多有限尺寸散射系统的有效介电常数参数这一艰巨任务。

PINN 代表了解决流体动力学、固体力学、电磁学和光学等各个领域复杂问题的变革性方法。它们的多功能性和效率继续推动研究和应用,使其成为未来计算建模工作的关键组成部分。

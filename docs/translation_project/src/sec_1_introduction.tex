\section{引言}

近几十年来,计算机技术的进步深刻地改变了科学研究的面貌。传统上,研究人员依靠理论推导和实验验证来研究自然现象。然而,计算方法的出现引入了各种数值模拟技术,使研究人员能够更深入地洞察复杂的现实世界系统。这些基于计算机的方法促进了对以前使用传统方法难以研究的复杂现象的探索和分析。因此,计算模拟已成为科学工具箱中必不可少的工具,增强了我们对不同学科的理解。

微分方程(DE)在各种科学和工程学科(Boussange 等 2023;Mallikarjunaiah 2023;Namaki 等 2023;Soldatenko 和 Yusupov 2017;Rodkina 和 Kelly 2011)中发挥着至关重要的作用,因为它们能够对涉及多个变量函数的复杂现象进行建模。它们的重要性源于许多物理过程,如热传导、流体动力学和波传播,都可以用捕捉函数及其导数之间关系的 PDE 来描述。根据自变量的数量,微分方程可分为以下两类(Taylor 等 2023):常微分方程(ODE)仅涉及一个自变量及其导数,
\begin{equation}
\frac{du}{dx} + u = 0.
\end{equation}
偏微分方程(PDE)涉及多个自变量及其偏导数,
\begin{equation}
\frac{\partial u}{\partial t} = k \frac{\partial^2 u}{\partial x^2}.
\end{equation}
与代数方程不同,微分方程表达了未知函数与其导数之间的方程关系。

在实际应用中,PDE 的求解在航空航天工程等领域变得极其重要,它们被用于优化飞机设计和分析结构周围的气流。此外,在气象学中,PDE 在天气预报模型中起着至关重要的作用,通过模拟大气动力学实现准确的预测。有效求解 PDE 的能力是各个学科推进技术和提高预测能力的基础。PDE 的数学理论已得到显著发展,产生了各种用于分析和求解 PDE 的方法。为了求解一些简化的偏微分方程,可以使用通用算子(Melchers 等 2023)。目前,求解偏微分方程较常用的方法有有限元法(FEM)(Liu 等 2022;Chernyshenko 和 Olshanskii 2015)、有限体积法(FVM)(Dick 2009)、基于粒子的方法(Oñate 和 Owen 2011)和有限单元法(FCM)(Kollmannsberger 等 2019)。FEM 已被证明是一种方便且准确的方法,利用了大量的计算资源(Innerberger 和 Praetorius 2023)。然而,多重迭代求解的实用性是有限的。为了增强 PDE 求解的能力,研究人员探索了样条函数的使用(Kolman 等 2017;Qin 等 2019)。同样,FEM 对 PDE 解进行离散化和数值近似,使得引入在表达 PDE 方面效率更高的傅里叶变换或拉普拉斯变换变得更加可行(Mugler 和 Scott 1988;El-Ajou 2021)(图 1)。

\begin{figure}[h!]
\centering
\includegraphics[width=0.9\textwidth]{images/Fig_1.png}
\caption{物理信息机器学习的各种应用}
\end{figure}

最近,深度学习的显著进步导致了计算机视觉和自然语言处理等各个领域的突破性发展。此外,深度学习对科学计算也产生了深远的影响。物理信息机器学习(PIML)作为一种结合传统机器学习技术与领域特定物理知识来解决复杂科学和工程问题的方法应运而生。与仅依赖数据的传统机器学习方法不同,PIML 将物理定律和约束整合到学习过程中,允许模型不仅从数据中学习,还可以从已知的控制方程、边界条件和其他物理原理中学习。这种混合方法增强了模型的泛化能力,特别是在数据稀缺、嘈杂或不完整的情况下。PIML 在各个领域发挥着越来越重要的作用,包括优化(Mowlavi 和 Nabi 2023;Hwang 等 2022;Zhang 等 2022b)、流体(Cai 等 2021;Di Leoni 等 2023;Zhang 等 2023;Ranade 等 2021;Sun 等 2020;Hanna 等 2022)、控制论(Schiassi 等 2022;Patel 等 2023;Zheng 等 2023;Faria 等 2024)、气候(Li 等 2024;Bonev 等 2023)、力学(Tripura 和 Chakraborty 2023;Wang 等 2023;Diao 等 2023;Le-Duc 等 2024)、物理学(Pan 等 2024;Liu 等 2024a;Jeong 等 2024;Jo 等 2024)、量子计算(Norambuena 等 2024;Xiao 等 2024;Sedykh 等 2024)等。

基于深度学习求解 PDE 问题的方法自 20 世纪 90 年代以来一直被持续研究(Dissanayake 和 Phan-Thien 1994;Lagaris 等 1998;Kumar 和 Yadav 2011;Blechschmidt 和 Ernst 2021;Hafiz 等 2024)。特别是,物理信息神经网络(PINN)作为处理正向和反向 PDE 问题的一种有效方法已获得突出的地位(Raissi 等 2019)。通过将控制 PDE 整合到损失函数中,PINN 利用自动微分来指导训练过程,确保学习到的模型遵守物理定律。这种无网格方法提供了极大的灵活性,使 PINN 能够在单个框架内无缝地结合基于物理的约束和观测数据。这种能力使得 PINN 在传统 PDE 求解器可能难以应对的场景中特别具有优势,例如高维问题或具有复杂几何形状的问题。此外,融合数据与物理模型的能力增强了预测的准确性和鲁棒性,从而实现更可靠的模拟和解决方案。随着研究的继续,PINN 的多功能性可能会扩展,进一步巩固其在解决各个领域具有挑战性的 PDE 问题中的作用(表 1)。

\subsection{物理信息神经网络 (PINN) 基础}

\subsubsection{问题表述}

PINN 是一类将通常由 PDE 描述的领域特定知识整合到神经网络训练过程中的机器学习方法。PINN 背后的核心思想是在神经网络训练期间将物理系统的控制方程作为约束强制执行(图 2)。

\begin{figure}[h!]
\centering
\includegraphics[width=0.9\textwidth]{images/Fig_2.png}
\caption{PINN 的框架}
\end{figure}

在求解 PDE 约束优化问题的背景下,解 $u(x, t)$ 代表状态变量,其中 $x$ 表示空间坐标,$t$ 表示时间。相关的 PDE 通常可以写为:
\begin{equation}
\begin{aligned}
f\left(x, t, \frac{\partial}{\partial x}u, \frac{\partial}{\partial t}u, \lambda\right) &= 0, \quad x \in \Omega, t \in [0, T], \\
u(x, 0) &= h(x), \quad x \in \Omega, \\
u(x, t) &= g(x, t), \quad x \in \partial\Omega, t \in [0, T],
(3)
\end{aligned}
\end{equation}
其中 $\partial$ 表示微分算子,$\lambda = [\lambda_1, \lambda_2, \cdots, \lambda_n]$ 表示 PDE 的参数,$f$ 表示 PDE 的残差。$\Omega$ 和 $\partial\Omega$ 分别表示求解域及其边界。初始条件由 $h(x)$ 给出,边界条件为 $g(x, t)$。

\begin{table}[htbp]
\centering
\caption{本文中使用的符号}
\label{tab:notations}
\begin{tabular}{ll}
\hline
符号 & 描述 \\
\hline
$u$ & PDE 问题的未知解 \\
$x$ & 系统域中的空间坐标 \\
$t$ & 时间变量,通常用于时变问题 \\
$\lambda$ & 由 PDE 控制的物理系统参数 \\
$\theta$ & 神经网络参数 \\
$\Omega$ & 物理系统的域 \\
$\mathcal{L}_f$ & 残差损失 \\
$\mathcal{L}_b$ & 边界条件损失 \\
$\mathcal{L}_i$ & 初始条件损失 \\
\hline
\end{tabular}
\end{table}

利用神经网络 $\tilde{u}(x, t; \theta)$ 来近似解 $u(x, t)$,其中 $\theta$ 表示神经网络参数。为了将此 PDE 整合到神经网络训练中,我们可以定义一个由两项组成的损失函数:强制神经网络拟合可用数据的数据驱动项,以及强制 PDE 约束的物理信息项。损失函数可以公式化为:
\begin{equation}
\mathcal{L} = \mathcal{L}_{\text{physics}} + \mathcal{L}_{\text{data}},
\end{equation}
这里,$\mathcal{L}_{\text{physics}}$ 强制执行如下 PDE 约束:
\begin{equation}
\mathcal{L}_{\text{physics}} = w_f \mathcal{L}_f(\theta; \mathcal{T}_f) + w_b \mathcal{L}_b(\theta; \mathcal{T}_b) + w_i \mathcal{L}_i(\theta; \mathcal{T}_i),
\end{equation}
其中
\begin{equation}
\begin{aligned}
\mathcal{L}_f(\theta; \mathcal{T}_f) &= \frac{1}{|\mathcal{T}_f|} \sum_{x_f \in \mathcal{T}_f} |f(x_f, t, \lambda)|^2, \\
\mathcal{L}_b(\theta; \mathcal{T}_b) &= \frac{1}{|\mathcal{T}_b|} \sum_{x_b \in \mathcal{T}_b} |\tilde{u}(x_b, t; \theta) - g(x_b, t)|^2, \\
\mathcal{L}_i(\theta; \mathcal{T}_i) &= \frac{1}{|\mathcal{T}_i|} \sum_{x_i \in \mathcal{T}_i} |\tilde{u}(x_i, 0; \theta) - h(x_i)|^2,
\end{aligned}
\end{equation}
其中 $w_f$、$w_b$ 和 $w_i$ 是权重;$\mathcal{T}_f$、$\mathcal{T}_b$、$\mathcal{T}_i$ 分别是域内部、边界条件和初始条件的样本集。$\mathcal{L}_{\text{data}}$ 衡量预测数据与观测数据之间的误差:
\begin{equation}
\mathcal{L}_{\text{data}} = \frac{1}{|\mathcal{T}_d|} \sum_{x_d \in \mathcal{T}_d} |\tilde{u}(x_d, t; \theta) - u(x_d, t)|^2,
\end{equation}
其中 $\{u(x_d, t)\}_{x_d \in \mathcal{T}_d}$ 表示已知的观测数据。通过使用梯度下降等技术最小化此组合损失函数,PINN 可以直接从数据中学习系统的底层物理机制,使其成为解决反问题和发现复杂物理系统中隐藏模式的有力工具。

\subsubsection{PINN 的性能评估}

PINN 的评估涉及几个关键指标,用于评估解的准确性以及网络遵守控制方程中嵌入的底层物理原理的程度。这些指标对于确定 PINN 在求解复杂 PDE 时的有效性至关重要,同时确保解保持物理一致性并遵守边界和初始条件。下面,我们详细阐述评估 PINN 性能时常用的关键评估指标。

\begin{itemize}
    \item \textbf{$L^2$ 相对误差} \\
    $L^2$ 相对误差衡量向量空间中的相对误差。在数学上,PINN 预测值 $\tilde{u}(x, t; \theta)$ 与真实解 $u(x, t)$ 之间的 $L^2$ 相对误差可以表示如下:
    \begin{equation}
    \text{L2 error} = \frac{\|\tilde{u}(x, t; \theta) - u(x, t)\|_2}{\|u(x, t)\|_2} = \frac{\sqrt{\sum_{i=1}^{N} |\tilde{u}(x_i, t_i; \theta) - u(x_i, t_i)|^2}}{\sqrt{\sum_{i=1}^{N} |u(x_i, t_i)|^2}},
    \end{equation}
    其中 $\|\cdot\|_2$ 表示 $L^2$ 范数。

    \item \textbf{均方根误差 (RMSE)} \\
    RMSE 作为量化预测误差离散度的关键指标。它是回归和机器学习模型中常用的指标,用于评估预测的准确性。在数学上,RMSE 定义为:
    \begin{equation}
    \text{RMSE} = \sqrt{\frac{1}{N} \sum_{i=1}^{N} |\tilde{u}(x_i, t_i; \theta) - u(x_i, t_i)|^2}
    \end{equation}

    \item \textbf{PDE 残差} \\
    PDE 残差衡量预测解满足 PDE 的程度。它表示模型的解在空间和时间上的给定点偏离满足控制方程的程度。在数学上,对于形式为 $\mathcal{L}(u(x, t)) = f(x, t)$ 的一般 PDE,其中 $\mathcal{L}$ 是微分算子,$f(x, t)$ 是 PDE 的源项。PDE 残差表示如下:
    \begin{equation}
    \text{PDE residual} = \frac{1}{n} \sum_{i=1}^{n} (\mathcal{L}(\tilde{u}(x_i, t_i; \theta)) - f(x_i, t_i))^2.
    \end{equation}
\end{itemize}

\subsection{论文组织和贡献}
本文的其余部分结构如下。第 2 节对现有的 PINN 基 PDE 求解器文献进行了全面分析。第 3 节展示了 PINN 的应用,第 4 节介绍了目前公开用于执行 PINN 的工具和代码。第 5 节讨论了该领域的挑战和一些可能的未来方向。最后,第 6 节总结了全文。

本综述旨在提供 PINN 的全面概述,重点介绍其理论基础、应用和进展。这项工作的主要贡献如下:
\begin{itemize}
    \item \textbf{PINN 简介}:本文清晰地介绍了 PINN 的基本原理,解释了它们如何将深度学习技术与物理定律相结合来解决 PDE 问题。
    \item \textbf{关键方法论讨论}:综述深入探讨了 PINN 中的各种方法论,包括架构、自适应数据重采样方法、损失函数设计、特征嵌入等。
    \item \textbf{应用回顾}:本文调查了 PINN 在流体动力学、材料科学和生物学等多个领域的广泛应用,展示了它们在应对现实世界挑战方面的多功能性。
    \item \textbf{最新进展}:本文总结了 PINN 研究的最新进展,包括计算精度和效率的提高、泛化能力。
    \item \textbf{新兴范式和未来方向}:重点介绍了 PINN 与算子学习等新兴范式的融合。此外,本文讨论了 PINN 目前面临的挑战,并提出了未来研究的潜在方向,以提高其适用性和性能。
\end{itemize}